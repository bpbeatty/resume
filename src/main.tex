%!TEX TS-program = xelatex
\documentclass[]{src/friggeri-cv}
\usepackage{afterpage}
\usepackage{hyperref}
\usepackage{color}
\usepackage{xcolor}
\usepackage{parskip}
\usepackage{progressbar}
% remove link decorations: https://tex.stackexchange.com/questions/823/remove-ugly-borders-around-clickable-cross-references-and-hyperlinks
\hypersetup{
    pdftitle={Brian Paul Beatty},
    pdfauthor={Brian Beatty},
    pdfsubject={Brian Beatty - Resume},
    pdfkeywords={aws,bpmn,c++,c,camunda,developer,devops,docker,git,java,jenkins,kafka,linux,os,podman,programming,python,resume,software,sysadmin,ublue},
    pdfborder=0 0 0,
    colorlinks=false,       % no lik border color
    allbordercolors=white    % white border color for all
}
\RequirePackage{xcolor}
\definecolor{pblue}{HTML}{0395DE}

\begin{document}
\header{Brian}{ P. Beatty}
    {}

% Fake text to add separator
\fcolorbox{white}{gray}{\parbox{\dimexpr\textwidth-2\fboxsep-2\fboxrule}{%
.....
}}

% In the aside, each new line forces a line break
\begin{aside}
  \section{Location}
    Papillion, NE, USA
    ~
  \section{Github}
    \href{https://github.com/bpbeatty}{https://github.com/bpbeatty}
    ~
  \section{Mail}
    \href{mailto:brian@27megahertz.com}{\textbf{brian@}27megahertz.com}
    ~
  \section{Programming Languages}
    \begin{tabular}{p{1.1cm} p{2.5cm}}
      \progressbar[width=1.1cm,filledcolor=green]{1.0} & {BASH}

      \progressbar[width=1.1cm,filledcolor=green]{0.75} & {C/C++}

      \progressbar[width=1.1cm,filledcolor=green]{0.7} & {FORTRAN}

      \progressbar[width=1.1cm,filledcolor=green]{0.9} & {Java}

      \progressbar[width=1.1cm,filledcolor=green]{0.85} & {Python}

    \end{tabular}
    ~
  \section{Personal Skills}
    \begin{tabular}{p{1.2cm} p{2.0cm}}
      \progressbar[width=1.1cm,filledcolor=blue]{1.0} & {Ownership}

      \progressbar[width=1.1cm,filledcolor=blue]{1.0} & {Collaboration}

      \progressbar[width=1.1cm,filledcolor=blue]{1.0} & {Organization}

    \end{tabular}
    ~
  \section{Projects}
    \item[\rightarrow]\href{https://www.northropgrumman.com/space/chasing-the-ionosphere}{Chasing the Ionosphere}
    \item[\rightarrow]\href{https://github.com/bpbeatty/bluefin}{bluefin}
    \item[\rightarrow]\href{https://github.com/bpbeatty/fedora-coreos-proxmox}{fedora-coreos-proxmox}
    \item[\rightarrow]\href{https://github.com/bpbeatty/resume}{resume}
    \item[\rightarrow]\href{https://github.com/bpbeatty/startingpoint}{startingpoint}
    ~
  \section{Technologies}
    \item[\rightarrow]{Ansible}
    \item[\rightarrow]{Docker/Podman}
    \item[\rightarrow]{GNURadio}
    \item[\rightarrow]{libvirt/qemu/kvm}
    \item[\rightarrow]{Linux}
    ~
\end{aside}

\section{Education}
\begin{entrylist}
  \entry
    {2014}
    {Bachelors in Science Electronics Engineering}
    {University of Nebraska Lincoln}
    {Strong focus with hands-on experience using industry standard technologies.}
\end{entrylist}

\section{Experience}
\begin{entrylist}
  \entry
    {Dec 2021 - Present}
    {Deep-Space Advanced Radar Capability}
    {Northrop Grumman}
    {\begin{itemize}
        \item Implement micro-service based architecture using Java Spring Boot framework.
        \item Code review and deliver feedback effectively to increase team productivity.
        \item Mentor Junior engineers and help onboard new team members.
    \end{itemize}}
  \entry
    {2021, 2022, 2023 - Present}
    {Edge Of Space}
    {Northrop Grumman}
    {\begin{itemize}
        \item Designed/Implemented protocols for launch and recovery activities.
        \item Trained interns and Junior engineers on launch and recovery activities.
        \item Launched a high altitude balloon in support of geolocation of radio signals at the edge of space.
        \item Lead software developer, customized open source software project to work with off the shelf parts in support of balloon beacon.
        \item Developed ground based software defined radio geolocation system.
    \end{itemize}}
  \entry
    {Jan 2021 - Dec 2021}
    {Protected Forward Comms HF}
    {Northrop Grumman}
    {\begin{itemize}
        \item Designed/Implemented 30W software defined radio capable of over the air transmissions, multiple waveforms implemented FM, SSB, BPSK.
        \item Performed Ionospheric scientific study to characterize link reciprocity in HF NVIS communication systems for DARPA.
        \item Obtained Amateur Radio License of Amateur Extra Class operator in support of project.
    \end{itemize}}
  \entry
    {Jan 2021 - Dec 2021}
    {Sagittarius Ground}
    {Northrop Grumman}
    {\begin{itemize}
        \item Provided software capability utilizing Air Force weather models to aid in mission planning of ground to satellite communication system.
        \item Designed/implemented REST service capable of calculating decibel loss for a ground terminal to satellite communication link utilizing ITU models and forecast weather data.
    \end{itemize}}
  \entry
    {Mar 2019 - Sep 2020}
    {AFWWEBS Project, SEMS III}
    {Northrop Grumman}
    {\begin{itemize}
        \item Lead technical contributor for the AFWWEBS Di2E CI/CD Pipeline from initial concept, design, implementation, to validation.
        \item The project took a single AFWWEBS repository, Weatherview, and implemented a CI/CD pipeline, from developer commit to artifact storage.
        \item Enabled moving 30+ repositories from on prem hardware to Amazon gov cloud. Enabled developers to work remotely and deploy software on Air Force systems.
        \item Lead technical contributor for the Skymap visualization project from initial concept, design, implementation, to validation.Lead technical contributor for the Skymap visualization project from initial concept, design, implementation, to validation.
        \item The project took data received from SWAFS and visualized the data in a web browser.
    \end{itemize}}
  \entry
    {Oct 2016 - Mar 2019}
    {SWAFS Project, SEMS III}
    {Northrop Grumman}
    {\begin{itemize}
        \item Lead technical contributor for the Skymap project from initial concept, design, implementation, to validation.
        \item The project took an academic level software package (SKYMAP) and integrated it with existing SWAFS capabilities (SNFT).
    \end{itemize}}
\end{entrylist}
\end{document}
